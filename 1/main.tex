\documentclass{article}
\usepackage{graphicx}
\usepackage{amsmath}
\usepackage{array}
\usepackage{fancyhdr}
\usepackage{amssymb}
\usepackage[shortlabels]{enumitem}

\pagestyle{fancy}
\fancyhead[L]{Banghao Chi}
\fancyhead[C]{Homework 1}
\fancyhead[R]{25th Jan}

\fancyfoot[C]{\thepage}

\renewcommand{\headrulewidth}{0.5pt}
\renewcommand{\footrulewidth}{0.5pt}

\begin{document}

\section*{Exercise 1}

If $A$ and $B$ are sets, show that
$$
(A \setminus B) \cup (B \setminus A) = (A \cup B) \setminus (A \cap B)
$$

\textbf{Solution:}

We consider two scenarios here: 1) Both $A$ and $B$ are non-empty sets; 2) At least one of them are empty sets.

\subsection*{1)}

If both sets are non-empty sets, we need to prove this by showing both sides contain the exact same elements:

From left to right:

\begin{align*}
& \forall x \in (A \setminus B) \cup (B \setminus A) \\
\Rightarrow & x \in A \setminus B \lor x \in B \setminus A \\
\Rightarrow & x \in (A \land x \notin B) \lor x \in (B \land x \notin A) \\
\Rightarrow & x \in (A \land x \notin A \cap B) \lor x \in (B \land x \notin A \cap B) \\
\Rightarrow & x \notin A \cap B \land (x \in A \lor x \in B) \\
\Rightarrow & x \notin A \cap B \land x \in (A \cup B) \\
\Rightarrow & x \in (A \cup B) \setminus (A \cap B)
\end{align*}

meaning that $(A \setminus B) \cup (B \setminus A) \subseteq (A \cup B) \setminus (A \cap B)$.

From right to left:

\begin{align*}
& \forall x \in (A \cup B) \setminus (A \cap B) \\
\Rightarrow & x \in A \cup B \land x \notin A \cap B \\
\Rightarrow & (x \in A \lor x \in B) \land x \notin A \cap B \\
\Rightarrow & (x \in A \land x \notin A \cap B) \lor (x \in B \land x \notin A \cap B) \\
\Rightarrow & (x \in A \land x \notin B) \lor (x \in B \land x \notin A) \\
\Rightarrow & x \in A \setminus B \lor x \in B \setminus A \\
\Rightarrow & x \in (A \setminus B) \cup x \in (B \setminus A)
\end{align*}

meaning that $(A \cup B) \setminus (A \cap B) \subseteq (A \setminus B) \cup (B \setminus A)$.

Therefore, we have shown that $(A \setminus B) \cup (B \setminus A) = (A \cup B) \setminus (A \cap B)$ when both sets are non-empty.

\subsection*{2)}

If at least one of them is empty, we can directly show that both sides are empty. Since $A$ or $B$ is empty, we suppose $A$ is empty.

\begin{itemize}
\item Since $A$ is empty, $A \setminus B$ is empty, and $B \setminus A$ is set $B$. Therefore, the left side is set $B$.
\item Since $A$ is empty, $A \cup B$ is set $B$, $A \cap B$ is empty, which means that $A \cup B \setminus A \cap B = B \setminus \emptyset = B$. Therefore, the right side is also set $B$.
\end{itemize}

\subsection*{Conclusion}

We have shown that both sides are the other non-empty set when at least one of them is empty, and the same elements when both are non-empty. Therefore, the equation holds.

\newpage

\section*{Exercise 2}

Show that if $f: A \to B$ and $E, F$ are subsets of $A$, then
$$
f(E \cup F) = f(E) \cup f(F)
$$
and
$$
f(E \cap F) \subseteq f(E) \cap f(F)
$$

\textbf{Solution:}

\subsection*{1)}

$$f(E \cup F) = f(E) \cup f(F)$$

From left to right:

\begin{itemize}
    \item $\forall y \in f(E \cup F)$, $\exists x$ such that $f(x) = y$, where $x \in E \cup F$, meaning that $x \in E \lor x \in F$
    \item Since $E$ and $F$ are subsets of $A$ and $x \in E \lor x \in F$, we get $f(x) \in f(E) \lor f(x) \in f(F)$, meaning that $f(x) = y \in f(E) \cup f(F)$
    \item Therefore, we successfully proved that $f(E \cup F) \subseteq f(E) \cup f(F)$
\end{itemize}

From right to left:

\begin{itemize}
    \item $\forall y \in f(E) \cup f(F) \Rightarrow y \in f(E) \lor y \in f(F)$
    \item $\Rightarrow \exists x \in E \lor x \in F$ such that $f(x) = y$
    \item $x \in E \lor x \in F \Rightarrow x \in E \cup F \Rightarrow f(x) = y \in f(E \cup F)$
    \item Therefore, we successfully proved that $f(E) \cup f(F) \subseteq f(E \cup F)$
\end{itemize}

Hence, the equation holds for $f(E \cup F) = f(E) \cup f(F)$.

\subsection*{2)}

$$f(E \cap F) \subseteq f(E) \cap f(F)$$

From left to right:

\begin{align*}
    & \forall y \in f(E \cap F), \exists x \in E \cap F \text{ s.t.} f(x) = y \\
    \Rightarrow & x \in E \land x \in F \\
    \Rightarrow & f(x) \in f(E) \land x \in f(F) \\
    \Rightarrow & f(x) = y \in f(E) \cap f(F)
\end{align*}

\section*{Exercise 3}

Let $f: A \to B$ and $g: B \to C$ be functions.
\begin{itemize}
    \item a) Show that if $g \circ f$ is injective, then $f$ is injective.
    \item b) Show that if $g \circ f$ is surjective, then $g$ is surjective.
\end{itemize}

\textbf{Solution:}

\subsection*{a)}

To prove $f$ is injective, we need to show that if $f(x_1) = f(x_2)$, then $x_1 = x_2$.

\begin{itemize}
    \item Suppose $\exists x_1, x_2 \in A$ such that $f(x_1) = f(x_2)$
    \item Then $g(f(x_1)) = g(f(x_2))$
    \item Since $g \circ f$ is injective
    \item We get $x_1 = x_2$
    \item Hence by definition, $f$ is injective.
\end{itemize}    

\subsection*{b)}

To prove $g$ is surjective, we need to show that $\forall y \in C, \exists x \in B$ such that $y = g(x)$

\begin{itemize}
    \item Since $g \circ f$ is surjective, $\forall y \in C, \exists a \in A$ such that $y = g(f(a))$
    \item $\Rightarrow \exists x = f(a) \in B$ such that $g(f(a)) = g(x) = y$
    \item Hence by definition, $g$ is surjective.
\end{itemize}

\newpage

\section*{Exercise 4}

Prove that $2^n < n! \forall n \geq 4, n \in N$ \\

\textbf{Solution:}



\end{document}