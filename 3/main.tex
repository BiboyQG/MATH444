\documentclass{article}
\usepackage{graphicx}
\usepackage{amsmath}
\usepackage{array}
\usepackage{fancyhdr}
\usepackage{amssymb}
\usepackage[shortlabels]{enumitem}

\pagestyle{fancy}
\fancyhead[L]{Banghao Chi}
\fancyhead[C]{Homework 3}
\fancyhead[R]{14th Feb}

\fancyfoot[C]{\thepage}

\renewcommand{\headrulewidth}{0.5pt}
\renewcommand{\footrulewidth}{0.5pt}

\begin{document}

\section*{Exercise 1}
Show that if $a, b \in \mathbb{R}$, then
\[\max(a,b) = \frac{1}{2}(a + b + |a-b|)\]
and
\[\min(a,b) = \frac{1}{2}(a + b - |a-b|).\]

\textbf{Solution:} \\

We prove the two identities by considering the cases when $a \geq b$ and when $a < b$. \\

\textbf{1):} $a \geq b$ \\

In this case:
\begin{align*}
\max(a,b) &= a \\
\min(a,b) &= b \\
|a-b| &= a-b \text{ (since $a \geq b$)}
\end{align*}

For the first identity:
\begin{align*}
\frac{1}{2}(a + b + |a-b|) &= \frac{1}{2}(a + b + (a-b)) \\
&= \frac{1}{2}(2a) \\
&= a \\
&= \max(a,b)
\end{align*}

For the second identity:
\begin{align*}
\frac{1}{2}(a + b - |a-b|) &= \frac{1}{2}(a + b - (a-b)) \\
&= \frac{1}{2}(2b) \\
&= b \\
&= \min(a,b)
\end{align*}

\textbf{2):} $a < b$ \\

In this case:
\begin{align*}
\max(a,b) &= b \\
\min(a,b) &= a \\
|a-b| &= -(a-b) = b-a \text{ (since $a < b$)}
\end{align*}

For the first identity:
\begin{align*}
\frac{1}{2}(a + b + |a-b|) &= \frac{1}{2}(a + b + (b-a)) \\
&= \frac{1}{2}(2b) \\
&= b \\
&= \max(a,b)
\end{align*}

For the second identity:
\begin{align*}
\frac{1}{2}(a + b - |a-b|) &= \frac{1}{2}(a + b - (b-a)) \\
&= \frac{1}{2}(2a) \\
&= a \\
&= \min(a,b)
\end{align*}

Therefore, we have proven that for all $a, b \in \mathbb{R}$:
\[\max(a,b) = \frac{1}{2}(a + b + |a-b|)\]
and
\[\min(a,b) = \frac{1}{2}(a + b - |a-b|)\]

\newpage

\section*{Exercise 2}
Let $S := \{1-\frac{(-1)^n}{n} : n \in \mathbb{N}\}$. Find $\inf S$ and $\sup S$. As always, prove your result. \\

\textbf{Solution:} \\



\newpage

\section*{Exercise 3}
Show that if nonempty sets $A, B \subseteq \mathbb{R}$ are bounded above, then $A \cup B$ is also bounded above, and we have
\[\sup(A \cup B) = \max(\sup A, \sup B).\]

\textbf{Solution:} \\



\newpage

\section*{Exercise 4}
Prove that if $\lim(x_n) = x$ and if $x > 0$, then there exists a natural number $M$ such that $x_n > 0$ for all $n \geq M$. \\

\textbf{Solution:} \\



\newpage

\section*{Exercise 5}
Show that
\[\lim\left(\sqrt{n^2 + 1} - n\right) = 0.\]

\textbf{Solution:} \\



\end{document}