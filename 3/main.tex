\documentclass{article}
\usepackage{graphicx}
\usepackage{amsmath}
\usepackage{array}
\usepackage{fancyhdr}
\usepackage{amssymb}
\usepackage[shortlabels]{enumitem}

\pagestyle{fancy}
\fancyhead[L]{Banghao Chi}
\fancyhead[C]{Homework 3}
\fancyhead[R]{14th Feb}

\fancyfoot[C]{\thepage}

\renewcommand{\headrulewidth}{0.5pt}
\renewcommand{\footrulewidth}{0.5pt}

\begin{document}

\section*{Exercise 1}
Show that if $a, b \in \mathbb{R}$, then
\[\max(a,b) = \frac{1}{2}(a + b + |a-b|)\]
and
\[\min(a,b) = \frac{1}{2}(a + b - |a-b|).\]

\textbf{Solution:} \\

We prove the two identities by considering the cases when $a \geq b$ and when $a < b$. \\

\textbf{1):} $a \geq b$ \\

In this case:
\begin{align*}
\max(a,b) &= a \\
\min(a,b) &= b \\
|a-b| &= a-b \text{ (since $a \geq b$)}
\end{align*}

For the first identity:
\begin{align*}
\frac{1}{2}(a + b + |a-b|) &= \frac{1}{2}(a + b + (a-b)) \\
&= \frac{1}{2}(2a) \\
&= a \\
&= \max(a,b)
\end{align*}

For the second identity:
\begin{align*}
\frac{1}{2}(a + b - |a-b|) &= \frac{1}{2}(a + b - (a-b)) \\
&= \frac{1}{2}(2b) \\
&= b \\
&= \min(a,b)
\end{align*}

\textbf{2):} $a < b$ \\

In this case:
\begin{align*}
\max(a,b) &= b \\
\min(a,b) &= a \\
|a-b| &= -(a-b) = b-a \text{ (since $a < b$)}
\end{align*}

For the first identity:
\begin{align*}
\frac{1}{2}(a + b + |a-b|) &= \frac{1}{2}(a + b + (b-a)) \\
&= \frac{1}{2}(2b) \\
&= b \\
&= \max(a,b)
\end{align*}

For the second identity:
\begin{align*}
\frac{1}{2}(a + b - |a-b|) &= \frac{1}{2}(a + b - (b-a)) \\
&= \frac{1}{2}(2a) \\
&= a \\
&= \min(a,b)
\end{align*}

Therefore, we have proven that for all $a, b \in \mathbb{R}$:
\[\max(a,b) = \frac{1}{2}(a + b + |a-b|)\]
and
\[\min(a,b) = \frac{1}{2}(a + b - |a-b|)\]

\newpage

\section*{Exercise 2}
Let $S := \{1-\frac{(-1)^n}{n} : n \in \mathbb{N}\}$. Find $\inf S$ and $\sup S$. As always, prove your result. \\

\textbf{Solution:} \\

\begin{itemize}
   \item \textbf{Infimum:}

For any $x \in S$:
\begin{itemize}
    \item If $n$ is even: $x = 1-\frac{1}{n} \geq 1-\frac{1}{2} = \frac{1}{2}$
    \item If $n$ is odd: $x = 1+\frac{1}{n} > 1 > \frac{1}{2}$
\end{itemize}
Therefore, $\frac{1}{2}$ is a lower bound. \\

Since $\frac{1}{2}$ is achieved at $n=2$, it is the greatest lower bound. \\

\item \textbf{Supremum:}

For any $x \in S$:
\begin{itemize}
    \item If $n$ is even: $x = 1-\frac{1}{n} < 1 < 2$
    \item If $n$ is odd: $x = 1+\frac{1}{n} \leq 1+1 = 2$
\end{itemize}
Therefore, 2 is an upper bound. \\

Since 2 is achieved at $n=1$, it is the least upper bound.
\end{itemize}

Thus, $\inf S = \frac{1}{2}$ and $\sup S = 2$.
\newpage

\section*{Exercise 3}
Show that if nonempty sets $A, B \subseteq \mathbb{R}$ are bounded above, then $A \cup B$ is also bounded above, and we have
\[\sup(A \cup B) = \max(\sup A, \sup B).\]

\textbf{Solution:} \\

Since there are two parts to this problem, we will solve them one by one. If nonempty sets $A, B \subseteq \mathbb{R}$ are bounded above,
\begin{itemize}
    \item Show that $A \cup B$ is bounded above
    \item Prove that $\sup(A \cup B) = \max(\sup A, \sup B)$
\end{itemize}

\begin{itemize}
\item 1) Since $A$ is bounded above, by definition of supremum, $\forall a \in A$, $a \leq \text{sup}A$. 

Similarly, since $B$ is bounded above, $\forall b \in B$, $b \leq \text{sup}B$

Let $S = \max(\text{sup}A, \text{sup}B)$. Then for any $x \in A \cup B$:
\begin{itemize}
    \item If $x \in A$, then $x \leq \text{sup}A \leq S$
    \item If $x \in B$, then $x \leq \text{sup}B \leq S$
\end{itemize}
Therefore, $A \cup B$ is bounded above by $S$.

\item 2) Consider the above $S$. Since we've shown that $S$ is an upper bound for $A \cup B$, we only need to show that it's the least upper bound.

Suppose $S_i$ is any upper bound for $A \cup B$.
Then $S_i$ is an upper bound for both $A$ and $B$.
Therefore, $\text{sup}A \leq S_i$ and $\text{sup}B \leq S_i$.
Thus, $\max(\text{sup}A, \text{sup}B) = S \leq S_i$.

Since $S_i$ is any upper bound for $A \cup B$, we conclude that $S = \sup(A \cup B)$.

Therefore, $\sup(A \cup B) = \max(\sup A, \sup B)$.

\end{itemize}

\newpage

\section*{Exercise 4}
Prove that if $\lim(x_n) = x$ and if $x > 0$, then there exists a natural number $M$ such that $x_n > 0$ for all $n \geq M$. \\

\textbf{Solution:} \\

By the definition of limit, for any $\epsilon > 0$, 
$\exists M \in \mathbb{N}$ such that $|x_n - x| < \epsilon$ for all $n \geq M$. \\

Let $\epsilon = \frac{x}{2}$. Since $x > 0$, $\epsilon > 0$. \\

Then we have:
$\exists M \in \mathbb{N}$ such that $|x_n - x| < \frac{x}{2}$ for all $n \geq M$. \\

Breaking down the absolute value inequality:
$$-\frac{x}{2} < x_n - x < \frac{x}{2} \text{ for all } n \geq M$$

Adding $x$ to all parts of the inequality:
$$\frac{x}{2} < x_n < \frac{3x}{2} \text{ for all } n \geq M$$

Since $x > 0$, we have $\frac{x}{2} > 0$. \\

Therefore, $x_n > \frac{x}{2} > 0$ for all $n \geq M$. \\

Thus, we have proven that there exists an $M \in \mathbb{N}$ such that $x_n > 0$ for all $n \geq M$.

\newpage

\section*{Exercise 5}
Show that
\[\lim\left(\sqrt{n^2 + 1} - n\right) = 0.\]

\textbf{Solution:} \\



\end{document}