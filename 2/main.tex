\documentclass{article}
\usepackage{graphicx}
\usepackage{amsmath}
\usepackage{array}
\usepackage{fancyhdr}
\usepackage{amssymb}
\usepackage[shortlabels]{enumitem}

\pagestyle{fancy}
\fancyhead[L]{Banghao Chi}
\fancyhead[C]{Homework 2}
\fancyhead[R]{6th Feb}

\fancyfoot[C]{\thepage}

\renewcommand{\headrulewidth}{0.5pt}
\renewcommand{\footrulewidth}{0.5pt}

\begin{document}

\section*{Exercise 1}
(15 points) Construct a bijection between $\mathbb{N}$ and the set of all odd integers greater than one hundred. A proof of the bijectivity is required. \\

\textbf{Solution:} \\

Define $f: \mathbb{N} \rightarrow \{x \in \mathbb{Z} : x \text{ is odd and } x > 100\}$ by:

$$f(n) = 2n + 101$$

\textbf{Proof of Bijectivity:}\\

We will prove that $f$ is both injective and surjective. \\

\noindent
1) \textbf{Proof of Injectivity:}
   \begin{itemize}
    \item Let $n_1, n_2 \in \mathbb{N}$ and suppose $f(n_1) = f(n_2)$
    \item Then $2n_1 + 101 = 2n_2 + 101$
    \item $\implies 2n_1 = 2n_2$
    \item $\implies n_1 = n_2$
    \item Therefore, $f$ is injective.
   \end{itemize}

\noindent
2) \textbf{Proof of Surjectivity:}
   \begin{itemize}
    \item Let $y$ be any odd integer greater than 100.
    \item Let $n = \frac{y - 101}{2}$. We need to show that:
    \begin{itemize}
    \item a) $n \in \mathbb{N}$
    \item b) $f(n) = y$
    \end{itemize}

   \item Since $y$ is odd and greater than 100, we can write $y = 2m + 101$ for some $m \in \mathbb{N}$
   \item Plugging $y$ into the expression of $n$, we get $n = m \in \mathbb{N}$, satisfying (a)
   
   \item And $f(n) = 2(\frac{y - 101}{2}) + 101 = y - 101 + 101 = y$, satisfying (b)

   \end{itemize}

Therefore, $f$ is both injective and surjective, hence bijective.

\newpage

\section*{Exercise 2}
(15 points) Use mathematical induction to prove that if the set $A$ has $n$ elements, then the power set $\mathcal{P}(A)$ has $2^n$ elements.

\textbf{Solution:} \\

We will prove this by mathematical induction on $n$. \\

\textbf{Base Case:} Let $n = 0$, meaning $A = \emptyset$.
The power set $\mathcal{P}(\emptyset)$ contains only one element: the empty set itself.
Therefore, $|\mathcal{P}(\emptyset)| = 1 = 2^0$, so the statement holds for $n = 0$. \\

\textbf{Inductive Hypothesis:} 
Assume the statement holds for some $k \geq 0$. That is, if $|A| = k$, then $|\mathcal{P}(A)| = 2^k$. \\

\textbf{Inductive Step:} 
Let $B$ be a set with $k+1$ elements. We need to prove that $|\mathcal{P}(B)| = 2^{k+1}$.
Choose any element $x \in B$ and let $A = B \setminus \{x\}$. Then $|A| = k$.

We can partition $\mathcal{P}(B)$ into two disjoint sets:
\begin{align*}
S_1 &= \{X \in \mathcal{P}(B) : x \notin X\} \\
S_2 &= \{X \in \mathcal{P}(B) : x \in X\}
\end{align*}

After observing the above, we can conclude that:
\begin{itemize}
\item $S_1$ is identical to $\mathcal{P}(A)$, so $|S_1| = 2^k$ by the inductive hypothesis.
\item $S_2$ can be obtained by adding $x$ to each set in $S_1$, so $|S_2| = |S_1| = 2^k$.
\item $S_1$ and $S_2$ are disjoint and their union is $\mathcal{P}(B)$.
\end{itemize}

Therefore:
\begin{align*}
|\mathcal{P}(B)| &= |S_1| + |S_2| \\
&= 2^k + 2^k \\
&= 2 \cdot 2^k \\
&= 2^{k+1}
\end{align*}

By the principle of mathematical induction, we conclude that if $|A| = n$, then $|\mathcal{P}(A)| = 2^n$ for all $n \geq 0$.

\newpage

\section*{Exercise 3}
(15 points) Prove that the collection $\mathcal{F}(\mathbb{N})$ of all finite subsets of $\mathbb{N}$ is countable. \\

\textbf{Solution:} \\

Define a function $f: \mathcal{F}(\mathbb{N}) \rightarrow \mathbb{N}$ as follows: \\

For any finite subset $A \in \mathcal{F}(\mathbb{N})$, let
\[f(A) = \sum_{n \in A} 2^n\]

To prove that $f$ is injective, we will show that if $f(A) = f(B)$, then $A = B$. \\

Suppose $f(A) = f(B)$. Then:
\[\sum_{n \in A} 2^n = \sum_{n \in B} 2^n\]

Assume, for the sake of contradiction, that $A \neq B$. Then there exists some number that is in one set but not the other. Let $m$ be the largest such number. \\

Without loss of generality, assume $m \in A$ and $m \notin B$. Then:
\[f(A) - f(B) = 2^m + \sum_{n \in A \setminus \{m\}} 2^n - \sum_{n \in B} 2^n\]

By our choice of $m$ as the largest element where $A$ and $B$ differ, we know that for all $n > m$, either $n$ is in both sets or in neither set. Therefore, these terms cancel out in the difference. \\

For all $n < m$, $\sum_{n \in A \setminus \{m\}} 2^n$ can be at most $2^m - 1$ (a property of geometric series), and $\sum_{n \in B} 2^n$ can be at least $1$ ($2^0 = 1$). \\

Therefore, the difference of the sums is at most $2^m - 1 - 1 = 2^m - 2$. \\

Therefore:
\[f(A) - f(B) \geq 2^m - (2^m - 2) = 2\]

This contradicts our assumption that $f(A) = f(B)$. Thus, we must have $A = B$, suggesting the injectivity of $f$. \\

Since $f$ is injective, this suggests that $|\mathcal{F}(\mathbb{N})| \leq |\mathbb{N}|$, further suggesting that $\mathcal{F}(\mathbb{N})$ is also countable by the definition of countability.

\newpage

\section*{Exercise 4}
(15 points) Let $K := \{s + t\sqrt{2} : s,t \in \mathbb{Q}\}$. Show that, if $x_1, x_2 \in K$, then $x_1 + x_2 \in K$ and $x_1x_2 \in K$.

\textbf{Solution:}

\newpage

\section*{Exercise 5}
(15 points) Prove that $(a + b)^2 \leq 2(a^2 + b^2)$ for all $a,b \in \mathbb{R}$. Show that equality holds if and only if $a = b$.

\textbf{Solution:}

\end{document}